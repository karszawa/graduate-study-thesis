\section{分散システム}\label{ux5206ux6563ux30b7ux30b9ux30c6ux30e0}

分散システムは、通信ネットワークにより結合された複数のプロセス上で、共通の目的の計算をするためのシステムである。
分散システムは通常は複数のコンピュータを用いて構成されるため、物理的な共有メモリを持たない。
さらにプロセス間のメッセージ通信には遅延があり、それがどれほどの大きさかもわからないため、ある2つのプロセスでそれぞれ実行されたイベントのうち、
どちらが物理的に先に実行されたものなのかは区別できない。
ここでいうイベントとは、プロセスが行う意味的にまとまった処理のことを指し、
たとえば、関数の実行開始・実行終了やプロセス間を行き交うメッセージの送信・受信などが当てはまる。

このようなモデルを用いる利点は以下のとおりである。

\subsection{分散計算のモデル}\label{ux5206ux6563ux8a08ux7b97ux306eux30e2ux30c7ux30eb}

\begin{itemize}
\tightlist
\item
  Interleaving Model
\item
  Happened Before Model
\item
  Potentially Causality Model
\end{itemize}

\section{分散アルゴリズム}\label{ux5206ux6563ux30a2ux30ebux30b4ux30eaux30baux30e0}

\begin{itemize}
\tightlist
\item
  分散アルゴリズムの説明
\end{itemize}

\subsection{分散アルゴリズムの評価}\label{ux5206ux6563ux30a2ux30ebux30b4ux30eaux30baux30e0ux306eux8a55ux4fa1}

\begin{itemize}
\tightlist
\item
  いろいろな尺度で評価出来る
\item
  それらを説明
\end{itemize}

\subsection{論理時計}\label{ux8ad6ux7406ux6642ux8a08}

\begin{itemize}
\tightlist
\item
  時刻を表現する方法も様々
\end{itemize}
